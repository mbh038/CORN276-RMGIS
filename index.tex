% Options for packages loaded elsewhere
\PassOptionsToPackage{unicode}{hyperref}
\PassOptionsToPackage{hyphens}{url}
\PassOptionsToPackage{dvipsnames,svgnames,x11names}{xcolor}
%
\documentclass[
  letterpaper,
  DIV=11,
  numbers=noendperiod]{scrreprt}

\usepackage{amsmath,amssymb}
\usepackage{iftex}
\ifPDFTeX
  \usepackage[T1]{fontenc}
  \usepackage[utf8]{inputenc}
  \usepackage{textcomp} % provide euro and other symbols
\else % if luatex or xetex
  \usepackage{unicode-math}
  \defaultfontfeatures{Scale=MatchLowercase}
  \defaultfontfeatures[\rmfamily]{Ligatures=TeX,Scale=1}
\fi
\usepackage{lmodern}
\ifPDFTeX\else  
    % xetex/luatex font selection
\fi
% Use upquote if available, for straight quotes in verbatim environments
\IfFileExists{upquote.sty}{\usepackage{upquote}}{}
\IfFileExists{microtype.sty}{% use microtype if available
  \usepackage[]{microtype}
  \UseMicrotypeSet[protrusion]{basicmath} % disable protrusion for tt fonts
}{}
\makeatletter
\@ifundefined{KOMAClassName}{% if non-KOMA class
  \IfFileExists{parskip.sty}{%
    \usepackage{parskip}
  }{% else
    \setlength{\parindent}{0pt}
    \setlength{\parskip}{6pt plus 2pt minus 1pt}}
}{% if KOMA class
  \KOMAoptions{parskip=half}}
\makeatother
\usepackage{xcolor}
\setlength{\emergencystretch}{3em} % prevent overfull lines
\setcounter{secnumdepth}{5}
% Make \paragraph and \subparagraph free-standing
\makeatletter
\ifx\paragraph\undefined\else
  \let\oldparagraph\paragraph
  \renewcommand{\paragraph}{
    \@ifstar
      \xxxParagraphStar
      \xxxParagraphNoStar
  }
  \newcommand{\xxxParagraphStar}[1]{\oldparagraph*{#1}\mbox{}}
  \newcommand{\xxxParagraphNoStar}[1]{\oldparagraph{#1}\mbox{}}
\fi
\ifx\subparagraph\undefined\else
  \let\oldsubparagraph\subparagraph
  \renewcommand{\subparagraph}{
    \@ifstar
      \xxxSubParagraphStar
      \xxxSubParagraphNoStar
  }
  \newcommand{\xxxSubParagraphStar}[1]{\oldsubparagraph*{#1}\mbox{}}
  \newcommand{\xxxSubParagraphNoStar}[1]{\oldsubparagraph{#1}\mbox{}}
\fi
\makeatother


\providecommand{\tightlist}{%
  \setlength{\itemsep}{0pt}\setlength{\parskip}{0pt}}\usepackage{longtable,booktabs,array}
\usepackage{calc} % for calculating minipage widths
% Correct order of tables after \paragraph or \subparagraph
\usepackage{etoolbox}
\makeatletter
\patchcmd\longtable{\par}{\if@noskipsec\mbox{}\fi\par}{}{}
\makeatother
% Allow footnotes in longtable head/foot
\IfFileExists{footnotehyper.sty}{\usepackage{footnotehyper}}{\usepackage{footnote}}
\makesavenoteenv{longtable}
\usepackage{graphicx}
\makeatletter
\newsavebox\pandoc@box
\newcommand*\pandocbounded[1]{% scales image to fit in text height/width
  \sbox\pandoc@box{#1}%
  \Gscale@div\@tempa{\textheight}{\dimexpr\ht\pandoc@box+\dp\pandoc@box\relax}%
  \Gscale@div\@tempb{\linewidth}{\wd\pandoc@box}%
  \ifdim\@tempb\p@<\@tempa\p@\let\@tempa\@tempb\fi% select the smaller of both
  \ifdim\@tempa\p@<\p@\scalebox{\@tempa}{\usebox\pandoc@box}%
  \else\usebox{\pandoc@box}%
  \fi%
}
% Set default figure placement to htbp
\def\fps@figure{htbp}
\makeatother
% definitions for citeproc citations
\NewDocumentCommand\citeproctext{}{}
\NewDocumentCommand\citeproc{mm}{%
  \begingroup\def\citeproctext{#2}\cite{#1}\endgroup}
\makeatletter
 % allow citations to break across lines
 \let\@cite@ofmt\@firstofone
 % avoid brackets around text for \cite:
 \def\@biblabel#1{}
 \def\@cite#1#2{{#1\if@tempswa , #2\fi}}
\makeatother
\newlength{\cslhangindent}
\setlength{\cslhangindent}{1.5em}
\newlength{\csllabelwidth}
\setlength{\csllabelwidth}{3em}
\newenvironment{CSLReferences}[2] % #1 hanging-indent, #2 entry-spacing
 {\begin{list}{}{%
  \setlength{\itemindent}{0pt}
  \setlength{\leftmargin}{0pt}
  \setlength{\parsep}{0pt}
  % turn on hanging indent if param 1 is 1
  \ifodd #1
   \setlength{\leftmargin}{\cslhangindent}
   \setlength{\itemindent}{-1\cslhangindent}
  \fi
  % set entry spacing
  \setlength{\itemsep}{#2\baselineskip}}}
 {\end{list}}
\usepackage{calc}
\newcommand{\CSLBlock}[1]{\hfill\break\parbox[t]{\linewidth}{\strut\ignorespaces#1\strut}}
\newcommand{\CSLLeftMargin}[1]{\parbox[t]{\csllabelwidth}{\strut#1\strut}}
\newcommand{\CSLRightInline}[1]{\parbox[t]{\linewidth - \csllabelwidth}{\strut#1\strut}}
\newcommand{\CSLIndent}[1]{\hspace{\cslhangindent}#1}

\usepackage{booktabs}
\usepackage{longtable}
\usepackage{array}
\usepackage{multirow}
\usepackage{wrapfig}
\usepackage{float}
\usepackage{colortbl}
\usepackage{pdflscape}
\usepackage{tabu}
\usepackage{threeparttable}
\usepackage{threeparttablex}
\usepackage[normalem]{ulem}
\usepackage{makecell}
\usepackage{xcolor}
\KOMAoption{captions}{tableheading}
\makeatletter
\@ifpackageloaded{tcolorbox}{}{\usepackage[skins,breakable]{tcolorbox}}
\@ifpackageloaded{fontawesome5}{}{\usepackage{fontawesome5}}
\definecolor{quarto-callout-color}{HTML}{909090}
\definecolor{quarto-callout-note-color}{HTML}{0758E5}
\definecolor{quarto-callout-important-color}{HTML}{CC1914}
\definecolor{quarto-callout-warning-color}{HTML}{EB9113}
\definecolor{quarto-callout-tip-color}{HTML}{00A047}
\definecolor{quarto-callout-caution-color}{HTML}{FC5300}
\definecolor{quarto-callout-color-frame}{HTML}{acacac}
\definecolor{quarto-callout-note-color-frame}{HTML}{4582ec}
\definecolor{quarto-callout-important-color-frame}{HTML}{d9534f}
\definecolor{quarto-callout-warning-color-frame}{HTML}{f0ad4e}
\definecolor{quarto-callout-tip-color-frame}{HTML}{02b875}
\definecolor{quarto-callout-caution-color-frame}{HTML}{fd7e14}
\makeatother
\makeatletter
\@ifpackageloaded{bookmark}{}{\usepackage{bookmark}}
\makeatother
\makeatletter
\@ifpackageloaded{caption}{}{\usepackage{caption}}
\AtBeginDocument{%
\ifdefined\contentsname
  \renewcommand*\contentsname{Table of contents}
\else
  \newcommand\contentsname{Table of contents}
\fi
\ifdefined\listfigurename
  \renewcommand*\listfigurename{List of Figures}
\else
  \newcommand\listfigurename{List of Figures}
\fi
\ifdefined\listtablename
  \renewcommand*\listtablename{List of Tables}
\else
  \newcommand\listtablename{List of Tables}
\fi
\ifdefined\figurename
  \renewcommand*\figurename{Figure}
\else
  \newcommand\figurename{Figure}
\fi
\ifdefined\tablename
  \renewcommand*\tablename{Table}
\else
  \newcommand\tablename{Table}
\fi
}
\@ifpackageloaded{float}{}{\usepackage{float}}
\floatstyle{ruled}
\@ifundefined{c@chapter}{\newfloat{codelisting}{h}{lop}}{\newfloat{codelisting}{h}{lop}[chapter]}
\floatname{codelisting}{Listing}
\newcommand*\listoflistings{\listof{codelisting}{List of Listings}}
\makeatother
\makeatletter
\makeatother
\makeatletter
\@ifpackageloaded{caption}{}{\usepackage{caption}}
\@ifpackageloaded{subcaption}{}{\usepackage{subcaption}}
\makeatother

\usepackage{bookmark}

\IfFileExists{xurl.sty}{\usepackage{xurl}}{} % add URL line breaks if available
\urlstyle{same} % disable monospaced font for URLs
\hypersetup{
  pdftitle={CORN276-RMGIS},
  pdfauthor={Michael Hunt},
  colorlinks=true,
  linkcolor={blue},
  filecolor={Maroon},
  citecolor={Blue},
  urlcolor={Blue},
  pdfcreator={LaTeX via pandoc}}


\title{CORN276-RMGIS}
\author{Michael Hunt}
\date{2026-02-06}

\begin{document}
\maketitle

\renewcommand*\contentsname{Table of contents}
{
\hypersetup{linkcolor=}
\setcounter{tocdepth}{2}
\tableofcontents
}

\bookmarksetup{startatroot}

\chapter*{Preface}\label{preface}
\addcontentsline{toc}{chapter}{Preface}

\markboth{Preface}{Preface}

This is a Quarto book.

To learn more about Quarto books visit
\url{https://quarto.org/docs/books}.

\bookmarksetup{startatroot}

\chapter{Visualising data}\label{visualising-data}

When we have got our data safely tucked into a spreadsheet. Now we need
to tease out of it the answers to our question(s) and to decide whether
we have evidence enough to reject our null hypotheses, or not, in which
case we will fail to reject them.

Let's take the example of the Palmer penguins dataset. This set contains
measurements of bill depth, bill length, flipper length and body mass of
males and females of three species of penguins: Adelie, Chinstrap and
Gentoo observed on any one of three islands in the Palmer Archipelago,
Antarctica.

\begin{figure}[H]

{\centering \pandocbounded{\includegraphics[keepaspectratio]{figures/penguins.png}}

}

\caption{Meet the penguins}

\end{figure}%

Let's consider only the females and ask the question:

\textbf{Question}: Is there any difference in body weight between the
three species:

from which we can generate a hypothesis:

\textbf{Hypothesis}: There is a difference in body weight between
females ofthe three species.

\textbf{Null hypothesis}: There is no difference in bosy weight between
females of the three species.

and hence a prediction of what we will find if the hypothesis is true:

\textbf{Prediction if the hypothesis is true}: At least one species will
have a different average body mass than at least one other species.

\section{Summarise the data}\label{summarise-the-data}

The first thing we can do to investigate our hypotheses is to summarise
the data. More often than not this means caclulating three things for
each sample - the sample sizes, the mean values and the standard errors
of those means.

\begin{tabular}[t]{l|r|r|r}
\hline
Species & N & Mean body mass (g) & Standard error (g)\\
\hline
Chinstrap & 34 & 3527.21 & 48.93\\
\hline
Gentoo & 58 & 4679.74 & 36.97\\
\hline
Adelie & 73 & 3368.84 & 31.53\\
\hline
\end{tabular}

\begin{tcolorbox}[enhanced jigsaw, toprule=.15mm, breakable, toptitle=1mm, opacityback=0, rightrule=.15mm, arc=.35mm, title=\textcolor{quarto-callout-note-color}{\faInfo}\hspace{0.5em}{Types of error bar}, colback=white, left=2mm, bottomtitle=1mm, leftrule=.75mm, titlerule=0mm, colframe=quarto-callout-note-color-frame, colbacktitle=quarto-callout-note-color!10!white, bottomrule=.15mm, coltitle=black, opacitybacktitle=0.6]

\textbf{standard deviations}: These tell us about the spread of values
in a sample or a population. They do not systematically get bigger or
smaller as the sample size increases. The standard deviation of a sample
can be used as an estimate of the standard deviation of the population.
We use standard deviations for \emph{descriptive purposes}

\textbf{standard errors of the mean} These are used to indicate how
precisely a sample mean estimates the true population mean. They are
used for \emph{inferential purposes}, whereby we try to infer from the
sample mean the range of values in which the true population mean might
be. Assuming normally distributed values, it would be very surprising if
the true population means were more than two standard errors away from
the sample means.

Standard errors are calculated from the standard deviations (SD) of the
sample using the formula \(\text{SE}=\frac{\text{SD}}{\sqrt{n}}\) where
\emph{n} is the sample size. This means that standard errors \emph{do}
get systematically smaller, the larger the sample. The larger the
sample, the closer the sample mean is likely to be to the true
population mean. Who knew?

\textbf{confidence intervals} These are also inferential tools. They
tell us the range of values within which the true mean might plausibly
lie, at some level of confidence, usually 95\%.

If you include error bars in a plot you can use any of these three
errors, depending on the story you want to tell. Whichever, just
\textbf{must} state in the figure caption which of them you have used.
Failing to do this can seriously mislead the reader, since they can be
of very different magnitudes.

\end{tcolorbox}

The errors calculated here are \textbf{standard errors of the mean}. We
use these because we want to get an idea, from our samples, of how
plausible it is that the population means differ from each other. These
population means could plausibly lie anywhere in the range that is our
sample means plus or minus two of these standard errors.

\begin{itemize}
\tightlist
\item
  Does it look as though there is evidence form the data for a
  difference between Adelie and Chinstrop penguins?
\item
  What about the Gentoos compared to either of the other two?
\item
  Do we have evidence to reject the null hypothesis. (Clue: yes we do!)
\end{itemize}

\section{Plot the data}\label{plot-the-data}

After summarising the data, then ext thing we nearly always do in
deciding what the data is telling us is to plot the data. We have
several choices of how to do so and each has its pros and cons. Let's
run through a few of them.

\subsection{Bar charts}\label{bar-charts}

\begin{figure}

\begin{minipage}{0.50\linewidth}

\centering{

\pandocbounded{\includegraphics[keepaspectratio]{visualising_data_files/figure-pdf/fig-bar-charts-1.pdf}}

}

\subcaption{\label{fig-bar-charts-1}Terrible!}

\end{minipage}%
%
\begin{minipage}{0.50\linewidth}

\centering{

\pandocbounded{\includegraphics[keepaspectratio]{visualising_data_files/figure-pdf/fig-bar-charts-2.pdf}}

}

\subcaption{\label{fig-bar-charts-2}Standard errors added}

\end{minipage}%
\newline
\begin{minipage}{0.50\linewidth}

\centering{

\pandocbounded{\includegraphics[keepaspectratio]{visualising_data_files/figure-pdf/fig-bar-charts-3.pdf}}

}

\subcaption{\label{fig-bar-charts-3}Sample sizes added}

\end{minipage}%
%
\begin{minipage}{0.50\linewidth}

\centering{

\pandocbounded{\includegraphics[keepaspectratio]{visualising_data_files/figure-pdf/fig-bar-charts-4.pdf}}

}

\subcaption{\label{fig-bar-charts-4}Redundant colours removed}

\end{minipage}%

\caption{\label{fig-bar-charts}Bar charts, from worst to best}

\end{figure}%

\section{Histograms}\label{histograms}

In histograms the range of a variable is split into bis of. certain
width, then the number of observations that fall within each bin is
displayed.

They can be used to inspect a data set, even one with multiple
categories, as with the penguin data. Unlike bar charts they do show the
distribution of the dataset, including its centtal value, spread and
symmetry, or lack thereof.

They do need care however in choice of the width of the bins. Make thm
too narrow and they can look gappy, with too much scatter ntrocued by
there not being many observations in each bin. Make the bins too wide
and much of the detail of the distribution is lost. You need to find,
approximately, the `Goldilocks' width, one that is just right.
Sometimes, though, you choose a binwidth that has meaning to you and the
reader, such as widths of 1 m/s if you were doing a histogram of a set
of wind speed measurements.

\begin{figure}

\begin{minipage}{0.33\linewidth}

\centering{

\pandocbounded{\includegraphics[keepaspectratio]{visualising_data_files/figure-pdf/fig-histograms-1.pdf}}

}

\subcaption{\label{fig-histograms-1}bins too narrow}

\end{minipage}%
%
\begin{minipage}{0.33\linewidth}

\centering{

\pandocbounded{\includegraphics[keepaspectratio]{visualising_data_files/figure-pdf/fig-histograms-2.pdf}}

}

\subcaption{\label{fig-histograms-2}Bin width about right}

\end{minipage}%
%
\begin{minipage}{0.33\linewidth}

\centering{

\pandocbounded{\includegraphics[keepaspectratio]{visualising_data_files/figure-pdf/fig-histograms-3.pdf}}

}

\subcaption{\label{fig-histograms-3}Bins too wide}

\end{minipage}%

\caption{\label{fig-histograms}Histograms of different bin width}

\end{figure}%

\pandocbounded{\includegraphics[keepaspectratio]{visualising_data_files/figure-pdf/unnamed-chunk-13-1.pdf}}

\pandocbounded{\includegraphics[keepaspectratio]{visualising_data_files/figure-pdf/unnamed-chunk-14-1.pdf}}

\bookmarksetup{startatroot}

\chapter*{References}\label{references}
\addcontentsline{toc}{chapter}{References}

\markboth{References}{References}

\phantomsection\label{refs}
\begin{CSLReferences}{0}{1}
\end{CSLReferences}




\end{document}
